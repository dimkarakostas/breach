\chapter{Εισαγωγή}\label{ch:intro}
\epigraph{\itshape Even if you're not doing anything wrong, you are being watched and
recorded.}{---Edward Snowden}

\section{Εισαγωγή}\label{sec:intro}

Το καλοκαίρι του 2013 επιβεβαιώθηκε αυτό που υπήρχε ως υποψία όλα τα προηγούμενα
χρόνια: οι συνομιλίες παρακολουθούνται και τα δεδομένα που ανταλλάσσονται μέσω
Διαδικτύου δεν είναι ασφαλή. Οι αποκαλύψεις Snowden άλλαξαν τον τρόπο με τον
οποίο αντιλαμβανόμαστε τη χρήση online υπηρεσιών και έστρεψαν πολλούς ερευνητές
και χρήστες στην αναζήτηση λύσεων ώστε οι επικοινωνίες να γίνουν πιο ασφαλείς
απέναντι σε κάθε είδους αντιπάλους.

Η παρούσα εργασία στοχεύει να αναδείξει αδυναμίες στα πρωτόκολλα που επιτρέπουν
την επικοινωνία μέσω Διαδικτύου και μέσω της δημοσίευσής της να κινητοποιήσει
την κοινότητα ώστε να αντιμετωπιστούν αυτά τα προβλήματα.

Η έρευνά μας επικεντρώνεται σε επιθέσεις που εκμεταλλεύονται τους αλγόριθμους
συμπίεσης που χρησιμοποιούνται πάνω στα δεδομένα που ανταλλάσσονται, προτού αυτά
κρυπτογραφηθούν και αποσταλούν. Συγκεκριμένα, επεκτείνουμε υπάρχοντα μο-ντέλα,
όπως το BREACH, ώστε να καταδείξουμε πως πρωτόκολλα τα οποία θεωρούνται σήμερα
απολύτως ασφαλή είναι πρακτικά τρωτά σε παρόμοιες επιθέσεις.

Κατά τη διάρκεια της έρευνάς μας επικεντρωθήκαμε στο λογισμικό συμπίεσης gzip,
το οποίο εφαρμόζει τον αλγόριθμο DEFLATE, ο οποίος με τη σειρά του αποτελεί
συνδυασμό των αλγορίθμων συμπίεσης Huffman και LZ77. Συγκεκριμένα, η επίθεση
εκμεταλλεύε-ται την ανάλυση που κάνει ο LZ77 πάνω στο καθαρό κείμενο, ενώ
αντίθετα η ύπαρξη συμπίεσης Huffman εμποδίζει την εκτέλεση. Παρότι δεν ελέγξαμε
άλλους αλγόριθμους ή εμπορικές εφαρμογές συμπίεσης, είναι αρκετά ασφαλές να
υποθέσουμε πως αλγόριθ-μοι που ακολουθούν όμοιες τεχνικές είναι δυνητικά στόχοι
για παρόμοιες επιθέσεις.

Το πιο διαδεδομένο πρωτόκολλο ανταλλαγής δεδομένων στο Διαδίκτυο είναι το HTTP
(Hyper-Text Transfer Protocol). Είναι ευρέως αποδεκτό πως δεδομένα που
στέλνονται μέσω απλού HTTP και δεν είναι κρυπτογραφημένα θα πρέπει να
θεωρούνται ανασφαλή ως προς την ακεραιότητα και την αυθεντικότητά τους. Το κενό
στην ασφάλεια που αφήνει το απλό HTTP ήρθε να συμπληρώσει αρχικά το SSL (Secure
Socket Layer) και στη συνέχεια το TLS (Transport Layer Security). Το TLS
εισάγεται ως ένα επίπεδο δικτύου πριν το επίπεδο εφαρμογής και επιβάλλει την
κρυπτογράφηση των δεδομένων πριν αυτά σταλούν στο Διαδίκτυο.

Οι αλγόριθμοι κρυπτογράφησης που χρησιμοποιούνται εν γένει μπορούν να χωριστούν
σε δύο κύριες κατηγορίες: ροής και δέσμης. Στην πρώτη περίπτωση, τα δεδομένα
κρυ-πτογραφούνται ως μια συνεχής ροή, ενώ στη δεύτερη περίπτωση χωρίζονται σε
δέσμες ίσου μεγέθους και κρυπτογραφείται κάθε δέσμη χωριστά. Σε περίπτωση που τα
δεδομέ-να δεν κατανέμονται με ακρίβεια σε δέσμες, εισάγεται τεχνητός θόρυβος
ώστε να επιτευχθεί το επιθυμητό μέγεθος.

Ο κυριότερος αλγόριθμος ροής είναι ο RC4, ο οποίος πλέον θεωρείται ανασφαλής και
αποφεύγεται η χρήση του. Από την άλλη πλευρά, ο πιο διαδεδομένος αλγόριθμος
δέσμης είναι ο AES, ο οποίος χρησιμοποιείται σε διάφορες παραλλαγές από την
πλειοψηφία των συστημάτων. Η χρήση αλγορίθμων ροής καθιστά την επίθεση που
περιγράφουμε πολύ ευκολότερη, καθώς μειώνεται η ύπαρξη θορύβου που μπορεί να
επηρεάσει τα αποτελέσματα. Ωστόσο, κατά τη διάρκεια της έρευνάς μας, διαπιστώσαμε πως η
χρήση του AES δεν εξασφαλίζει απόλυτη ασφάλεια και υπό προϋποθέσεις είναι δυνατό
δεδομέ-να που ανταλλάσσονται με αυτές τις μεθόδους να υποκλαπούν.

Για να το επιτύχουμε αυτό έπρεπε αρχικά να μοντελοποιήσουμε την επίθεσή μας. Για
το σκοπό αυτό ορίσαμε μια νέα κρυπτογραφική ιδιότητα, την οποία ονομάζουμε
μη-διακρισιμότητα ενάντια σε επιθέσεις μερικώς επιλεγμένου κειμένου (IND-PCPA). Όμοιες
ιδιότητες, όπως IND-CPA, IND-CCA κ.ά, είναι ορισμένες στη βιβλιογραφία και
χρησιμοποιούνται ευρέως στην ανάλυση κρυπτοσυστημάτων. Η εισαγωγή της IND-PCPA
στοχεύει στην επέκταση των αναλύσεων ώστε να καλύπτουν επιθέσεις όπως αυτή που
αναπτύσσεται στην παρούσα εργασία.

Η επιτυχία της επίθεσης προϋποθέτει το σύστημα το οποίο αναλύεται να παρουσιάζει
συγκεκριμένα χαρακτηριστικά-παθογένειες. Η επίλυση των παθογενειών είναι
δεδομέ-νο πως βοηθάει σε σημαντικό βαθμό στην αντιμετώπιση της επίθεσης. Συνεπώς,
είναι σημαντικό να μοντελοποιήσουμε την επίθεση και να ορίσουμε τα
χαρακτηριστικά της, προτού επιχειρήσουμε να βρούμε τρόπους αντιμετώπισής της.

Η επίθεση που ερευνάται είναι επέκταση γνωστών μοντέλων, όπως αναφέρθηκε. Ωστό-σο
η ανάλυσή μας οδηγεί σε χαλάρωση των απαιτήσεων που θεωρούνταν δεδομένες και,
κατά συνέπεια, στοχεύει σε μεγαλύτερο εύρος συστημάτων. Είναι εμφανές πως
σε οποιοδήποτε σύστημα ικανοποιούνται οι απαιτήσεις που ορίζουμε η επίθεση είναι
δυνητικά εφικτή, συνεπώς το σύστημα θα πρέπει να θεωρείται ανασφαλές.

Στην παρούσα εργασία περιγράφονται αδυναμίες σε δύο εφαρμογές που
χρησιμοποιού-νται από μεγάλο ποσοστό χρηστών του Διαδικτύου. Η πρώτη είναι η
υπηρεσία chat του Facebook, όπου αναλύουμε τον τρόπο με τον οποίο προσωπικά
μηνύματα κάποιου χρήστη μπορούν να υποκλαπούν. Η δεύτερη είναι η υπηρεσία email
της Google, το Gmail. Σε αυτή την περίπτωση, παρουσιάζουμε πώς μπορεί κάποιος
επιτιθέμενος να αποκτήσει τον έλεγχο του λογαριασμού ενός χρήστη ώστε να είναι
σε θέση να υποδυθεί τον χρήστη, καθώς και να υποκλέψει δεδομένα που
ανταλλάχθησαν μέσω mail.

Για την εκτέλεση των πειραμάτων αναπτύξαμε λογισμικό σε επίπεδο
proof-of-concept, το οποίο μπορεί να χρησιμοποιηθεί για την εκτέλεση της
επίθεσης στα συγκεκριμένα συστήματα. Ωστόσο κάθε σύστημα παρουσιάζει
ιδιομορφίες, συνεπώς για να χρησιμο-ποιηθεί το ίδιο λογισμικό για την ανάλυση
άλλων συστημάτων θα πρέπει να προηγηθούν οι κατάλληλες τροποποιήσεις.

Σε αυτό το σημείο είναι σημαντικό να επικεντρωθούμε στο στατιστικό κομμάτι της
επίθεσης. Η επίθεση δεν μπορεί να θεωρηθεί ντετερμινιστική, καθώς η ανάλυσή μας
βασίζεται στη χρήση πιθανοτήτων. Είναι εμφανές ωστόσο πως στο βαθμό που
εξασφα-λίζουμε μεγαλύτερη εμπιστοσύνη και μειώνουμε το στατιστικό λάθος, τα
αποτελέσματα είναι δυνατό να προκύψουν σε λιγότερο χρόνο και με μεγαλύτερη
ακρίβεια.

Ο πιθανοτικός παράγοντας της επίθεσης μας οδήγησε στην ανάπτυξη μεθόδων
βελτι-στοποίησης. Ο στόχος μας αφορά σε δύο κατευθύνσεις: μείωση των στατιστικών
δειγμά-των και ελαχιστοποίηση του χρόνου συλλογής κάθε δείγματος.

Στην πρώτη περίπτωση είναι αναγκαίο να οριστεί ένα κατάλληλο πλήθος δειγμάτων,
τα οποία οδηγούν σε ένα ασφαλές συμπέρασμα. Βάσει του νόμου των μεγάλων αριθμών,
όσο περισσότερα δείγματα συλλέγουμε τόσο καλύτερα αποτελέσματα αναμένουμε.
Ωστόσο από ένα σημείο και μετά ο χρόνος εκτέλεσης καθιστά μεγαλύτερο πλήθος
δειγμάτων απαγορευτικό. Για αυτό το λόγο αναλύσαμε την στατιστική κατανομή του
θορύβου και καταλήξαμε σε συγκεκριμένο πλήθος δειγμάτων από το οποίο μπορούν να
προκύψουν αξιόπιστα αποτελέσματα για κάθε περίπτωση.

Στη δεύτερη περίπτωση ερευνήσαμε τη λειτουργία των προγραμμάτων περιήγησης του
Διαδικτύο (browsers) και των προτοκόλλων των επιπέδων μεταφοράς και δικτύου.
Δημιουργήσαμε τεχνικές παραλληλοποίησης, οι οποίες επιτρέπουν τη διαίρεση των
αναγκαίων δειγμάτων με αποδοτικές μεθόδους και τη συλλογή τους από πολλές πηγές
ταυτόχρονα. Εν τέλει, κάθε τεχνική μπορεί να οδηγήσει σε επιτάχυνση της επίθεσης
κατά αρκετές τάξεις μεγέθους.

Τα αποτελέσματα που προέκυψαν για τις υπηρεσίες που ελέγξαμε μπορούν να
θεωρη-θούν επιτυχημένα. Συγκεκριμένα, αποδείξαμε ότι οι αδυναμίες που βρήκαμε
μπορούν να χρησιμοποιηθούν όπως αναμέναμε και καταφέραμε να υποκλέψουμε
τουλάχιστον ένα byte δεδομένων σε κάθε περίπτωση. Ωστόσο, ο χρόνος που
απαιτείται για την ολοκλήρωση της επίθεσης είναι της τάξης των εβδομάδων ή
μηνών, συνεπώς, ανάλογα με τις απαιτήσεις του επιτιθέμενου, η επίθεση μπορεί να
θεωρηθεί μη-ρεαλιστική. Σε κάθε περίπτωση, τα αποτελέσματά μας καταδεικνύουν ότι
τα συστήματα που αναλύσα-με, στο βαθμό και υπό τις προϋποθέσεις που περιγράψαμε,
θα πρέπει να θεωρούνται ανασφαλή.

Η αντιμετώπιση της επίθεσης θα πρέπει να αποτελέσει αντικείμενο μελέτης και να
υλοποιηθεί το συντομότερο δυνατόν. Η φύση της επίθεσης επιτρέπει επιλεκτικές
λύσεις, οι οποίες βελτιώνουν την ασφάλεια υπό προϋποθέσεις. Ωστόσο είναι
απαραίτητο να υλοποιηθούν πρότυπα τα οποία επικεντρώνονται στα δομικά προβλήματα
που επιτρέ-πουν τέτοιου είδους επιθέσεις και αντιμετωπίζουν ολοκληρωτικά τις
παθογένειες.

Στη βιβλιογραφία μπορούν να βρεθούν πλήθος προτάσεων που ως ένα βαθμό οδηγούν
σε βελτίωση της ασφάλειας των συστημάτων. Στην παρούσα εργασία αναλύουμε αρκε-τές
τέτοιες προτάσεις και εξηγούμε για ποιο λόγο δεν αποτελούν ριζική αντιμετώπιση
του προβλήματος. Στη συνέχεια, παρουσιάζουμε πρότυπα τα οποία εφόσον
υλοποιη-θούν είναι δυνατόν να εξαλείψουν ολοκληρωτικά επιθέσεις όπως αυτή που
ερευνήσαμε.

Εν τέλει, η παρούσα εργασία αποτελεί τη συνέχεια μια ομάδας ερευνών που
παρουσιά-στηκαν τα τελευταία χρόνια και φανέρωσαν βασικές αδυναμίες στα
συστήματα που χρησιμοποιούμε κατά κόρον. Είναι σημαντικό να επεκταθεί με νέες
τεχνικές βελτιστο-ποίησης της επίθεσης και, κυρίως, νέες μεθόδους αντιμετώπισής
της.

\section{Δομή της εργασίας}\label{sec:structure}

Η εργασία έχει δομηθεί ως εξής:

\begin{description} \item{Κεφάλαιο \ref{background}} \hfill \\

Το κεφάλαιο αυτό παρέχει στον αναγνώστη βασικές πληροφορίες, τόσο σε τεχνικό όσο
και σε θεωρητικό επίπεδο, οι οποίες θα χρησιμοποιηθούν στη συνέχεια.
Θα περιγράψουμε τους πιο διαδεδομένους αλγόριθμους συμπίεσης, καθώς και βασικά
πρωτοκόλλα που χρησιμοποιούνται για την ασφάλεια στις επικοινωνίες, καθώς και
επιθέσεις εναντίων τους.\hfill \\

\item{Κεφάλαιο \ref{ch:pcpa}} \hfill \\

Εισάγουμε μια νέα ιδιότητα για κρυπτοσυστήματα, περιγράφοντας αυστηρούς ορισμούς
για αυτήν. Τη συγκρίνουμε με γνωστές ιδιότητες κρυπτοσυστημάτων και
παρουσιάζουμε σενάρια επιθέσεων με βάση το νέο σχήμα.\hfill \\

\item{Κεφάλαιο \ref{ch:attack}} \hfill \\

Περιγράφουμε σε βάθος το μοντέλο επίθεσης που ερευνάται σε αυτή την εργασία.
Αναλύουμε την υλοποίησή μας για την επίθεση, παρουσιάζουμε παθογένειες σε μεγάλα
συστήματα, καθώς και μεθοδολογία ώστε να μπορεί να επιβεβαιωθεί κατά πόσο η
επίθεση είναι δυνατή όσον αφορά κάποιο συγκεκριμένο στόχο.\hfill \\

\item{Κεφάλαιο \ref{ch:statistic}} \hfill \\

Το κεφάλαιο αυτό περιέχει στατιστικές μεθόδους που χρησιμοποιήθηκαν κατά την
έρευνά μας. Προτείνονται πιθανοτικές τεχνικές ώστε να παρακαμφθούν εμπόδια,
καθώς και αρκετοί μηχανισμοί βελτιστοποίησης της επίθεσης.\hfill \\

\item{Κεφάλαιο \ref{ch:experiment}} \hfill \\

Παρουσιάζουμε τα αποτελέσματα εκτενών πειραμάτων σε ευρέως χρησιμοποιούμενα
συστήματα. Ορίζουμε τις πιθανότητες επιτυχίας της επίθεσης και παρουσιάζουμε
διαγράμματα απόδοσης για κάθε περίπτωση.\hfill \\

\item{Κεφάλαιο \ref{ch:mitigation}} \hfill \\

Περιγράφουμε μηχανισμούν αντιμετώπισης της επίθεσης. Αναλύουμε την απόδοση
παλαιών προτάσεων υπό το πρίσμα των δεδομένων που προέκυψαν από την παρούσα
εργασία και προτείνουμε καινοτόμες τεχνικές που θα μπορούσαν δυνητικά να
εξαλείψουν την επίθεση.\hfill \\

\item{Κεφάλαιο \ref{conclusion}} \hfill \\

Συμπυκνώνουμε τα αποτελέσματά μας και προτείνουμε πεδία έρευνας που θα μπορούσαν
μελλοντικά να βελτιώσουν το μοντέλο επίθεσης και να ελαχιστοποιήσουν της
συνέπειες.\hfill \\

\item{Κεφάλαιο \ref{ch:appendix}} \hfill \\

Ο κώδικας υλοποίησης της επίθεσης.  \end{description}

