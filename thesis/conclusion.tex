\chapter{Conclusion}\label{conclusion}

\section{Concluding remarks}

Attacks on encrypted protocols, that exploit compression methods applied on the
plaintext handled by those protocols, such as BREACH, have only recently been
described.  Literature works so far show limited theoretical definitions of this
new type of attacks, while experimental results touch a relatively small scope
of protocols used nowadays.

This work focused on assessing the threat of such attacks for widely used
protocols, expanding the theoretical definition, as well as investigated the
success of methods designed for mitigation.

We introduced a cryptographical game for determining the property of
indistinguishability under partially chosen plaintext attacks. Also, we provided
intuitive proofs for comparison to other indistinguishability properties, along
with scenarios of application of partially chosen plaintext attacks on
compressed encrypted protocols.

The need for practical description of our method resulted in the definition of
an attack model, based on BREACH, that initiates, automates and validates the
attack. We also revealed major vulnerabilities on the two systems that we
experimented on, Facebook and Gmail, introducing new forms of secrets and chosen
plaintext an attacker could use.

In advance, we expanded the scope of the attack to block ciphers, we ulitized
various statistical methods, that bypass known obstacles, such as noise and
padding.  Furthermore, we proposed various optimization techniques that could
reduce the time and increase the efficiency of the attack, posing a valid threat
for real-world systems.

In order to perform experiments and validate the efficiency of the attack, we
implemented a framework in Python, that initiates the attack on a chosen
endpoint and parses the output in order to produce statistical results. From an
attacker's perspective, the framework must run on a machine inside the victim's
network, while the victim's machine is configured to send all traffic to the
endpoint to the attacker's machine and the victim also  browses a website
controlled by the attacker.

Experimental results have shown that, although the framework does not provide a
robust, bulletproof functionality, the attacker has a considerable advantage on
stealing a secret from the endpoints tested.

Finally, we investigated the ability of previously proposed mitigation
techniques to stop the attack under the findings, as well as proposed novel
methods that could effectively minimize the attack's success or even mitigate it
completely.

\section{Future Work}

Although this paper introduced the IND-PCPA property, formal definitions and
mathematical proofs should also be used to properly describe it. Also, this new
property should be formally evaluated, compared to other known properties.

As far as the practical attack is concerned, a consistency mechanism, as
described in Section \ref{sec:persistence}, is needed, in order to take full
advantage of vulnerabilities of simple HTTP connections. Furthermore, the
integration of MitM attacks, as the ones referenced in Section \ref{sec:mitm},
would result in a potential threat outside lab environment.

Finally, especially in light of the new findings, implementation of the two
novel mitigation techniques, compressibility annotation
[\ref{subsec:annotation}] and SOS headers [\ref{subsec:sos}], is vital in order
for the systems to remain as secure as possible.
