\chapter{Partially Chosen Plaintext Attack}\label{ch:pcpa}

Traditionally, cryptographers have used games for security analysis. Such games
include the indistinguishability under chosen-plaintext-attack (IND-CPA), the
indistinguishability under chosen ciphertext attack/adaptive chosen ciphertext
attack (IND-CCA1, IND-CCA2) etc
\footnote{\url{https://en.wikipedia.org/wiki/Ciphertext_indistinguishability}}.
In this chapter, we introduce a definition for a new property of encryption
schemes, called indistinguishability under partially-chosen-plaintext-attack
(IND-PCPA). We will also show provide comparison between IND-PCPA and other
known forms of cryptosystem properties.

\section{Partially Chosen Plaintext Indistinguishability}\label{sec:indpcpa}

\subsection{Definition} IND-PCPA uses a definition similar to that of IND-CPA.
For a probabilistic asymmetric key encryption algorithm, indistinguishability
under partially chosen plaintext attack (IND-PCPA) is defined by the following
game between an adversary and a challenger.

\begin{itemize} \item The challenger generates a pair \begin{math}P_k,
S_k\end{math} and publishes \begin{math}P_k\end{math} to the adversary.  \item
The adversary may perform a polynomially bounded number of encryptions or other
operations.  \item Eventually, the adversary submits two distinct chosen
plaintexts \begin{math}M_0, M_1\end{math} to the challenger.  \item The
challenger selects a bit \begin{math}b\in{0, 1}\end{math} uniformly at random.
\item The adversary can then submit any number of selected plaintexts
\begin{math}R_i, i\in N, |R| \geq 0\end{math}, so the challenger sends the
ciphertext \begin{math}C_i = E(P_k, M_b||R_i)\end{math} back to the adversary.
\item The adversary is free to perform any number of additional computations or
encryptions, before finally guessing the value of \begin{math}b\end{math}.
\end{itemize}

A cryptosystem is indistinguishable under partially chosen plaintext attack, if
every probabilistic polynomial time adversary has only a negligible advantage on
finding \begin{math}b\end{math} over random guessing. An adversary is said to
have a negligible advantage if it wins the above game with probability
\begin{math}\frac{1}{2} + \epsilon(k)\end{math}, where
\begin{math}\epsilon(k)\end{math} is a negligible function in the security
parameter \begin{math}k\end{math}.

Intuitively, we can think that the adversary has the ability to modify the
plaintext of a message, by appending a portion of data of his own choice to it,
without knowledge of the plaintext itself. He can then acquire the ciphertext of
the modified text and perform any kinds of computations on it. A system could
then be described as IND-PCPA, if the adversary is unable to gain more
information about the plaintext, than he could by guessing at random.

\subsection{IND-PCPA vs IND-CPA}

Suppose the adversary submits the empty string as the chosen plaintext, a choice
which is allowed by the definition of the game. The challenger would then send
back the ciphertext \begin{math}C_i = E(P_k, M_b||"\ ") = E(P_k, M_b)\end{math},
which is the ciphertext returned from the challenger during the IND-CPA game.
Therefore, if the adversary has the ability to beat the game of IND-PCPA, i.e. if
the system is not indistinguishable under partially chosen plaintext attacks, he
also has the ability to beat the game of IND-CPA. Thus we have shown that
IND-PCPA is at least as strong as IND-CPA.

\section{PCPA on compressed encrypted protocols}\label{sec:cepcpa}

\subsection{Compression-before-encryption and vice versa} When having a system
that applies both compression and encryption on a given plaintext, it would be
interesting to investigate the order the transformations should be executed.

Lossless data compression algorithms rely on statistical patters to reduce the
size of the data to be compressed, without losing information. Such a method is
possible, since most real-world data has statistical redundancy. However, it can
be understood from the above that such compression algorithms will fail to
compress some data sets, if there is no statistical pattern to exploit.

Encryption algorithms rely on adding entropy on the ciphertext produced. If the
ciphertext contains repeated portions or statistical patterns, such behaviour
can be exploited to deduce the plaintext.

In the case that we apply compression after encryption, the text to be
compressed should demostrate no statistical analysis exploits, as described
above. That way compression will be unable to reduce the size of the data. In
addition, compression after encryption does not increase the security of the
protocol.

On the other hand, applying encryption after compression seems a better
solution. The compression algorithm can exploit the statistical redundancies of
the plaintext, while the encryption algorithm, if applied perfectly on the
compressed text, should produce a random stream of data. Also, since compression
also adds entropy, this scheme should make it harder for attackers who rely on
differential cryptanalysis to break the system.

\subsection{PCPA scenario on compression-before-encryption protocol}

Let's assume a system that composes encryption and compression in the following
manner:

\begin{math}c = Encrypt(Compress(m))\end{math}

where \begin{math}c\end{math} is the ciphertext and \begin{math}m\end{math} is
the plaintext.

Suppose the plaintext contains a specific secret, among random strings of data,
and the attacker can issue a PCPA with a chosen plaintext, which we will call
reflection. The plaintext then takes the form:

\begin{math}m = n_1 || secret || n_2 || reflection || n_3\end{math}

where \begin{math}n_1, n_2, n_3\end{math} are random nonces.

If the reflection is equal to the secret, the compression mechanism will
recognize the pattern and compress the two portions. In other case, the two
strings will not demonstrate any statistical redundancy and compression will
perform worse. As a result, in the first case the data to be encrypted will be
smaller than in the second case.

Most commonly encryption is done by a stream or a block cipher. In the first
case, the lengths of a plaintext and the corresponding ciphertext are identical,
whereas in the second case they differ by the number of the padding bits, which
is relatively small. That way, in the above case, an adversary could identify a
pattern and extract information about the plaintext, based on the lengths of the
two ciphertexts.

\section{Known PCPA exploits}\label{sec:known_pcpa}

\subsection{CRIME}

"Compression Ratio Info-leak Made Easy" (CRIME) \cite{crime} is a security exploit
that was revealed at the 2012 \href{https://www.ekoparty.org}{ekoparty}. As
described, "it decrypts HTTPS traffic to steal cookies and hijack sessions".

In order for the attack to succeed, there are two requirements. Firstly, the
attacker should be able to sniff the victim's network, so as to see the
request/response packet lengths. Secondly, the victim should visit a website
controlled by the attacker or surf on non-HTTPS sites, in order for the CRIME
JavaScript to be executed.

If the above requirements are met, the attacker makes a guess for the secret to
be stolen and asks the browser to send a request with this guess as the path.
The attacker can then observe the length of the request and, if the length is
less than usual, it is assumed that the guess string was compressed with the
secret, so it was correct.

CRIME has been mitigated by disabling TLS and SPDY compression on both Chrome
and Firefox browsers, as well as various server software packages. However, HTTP
compression is still supported, while some web servers that still support TLS
compression are also vulnerable.

\subsection{BREACH}

"Browser Reconnaissance and Exfiltration via Adaptive Compression of Hypertext"
(BREACH) \cite{breach} is a security exploit built based on CRIME. Presented at
the August 2013 \href{https://www.blackhat.com}{Black Hat} conference, it
targets the size of compressed HTTP responses and extracts secrets hidden in the
response body.

Like the CRIME attack, the attacker needs to sniff the victim's network traffic,
as well as force the victim's browser to issue requests on the chosen endpoint.
Additionally, the original attack works against stream ciphers only, it assumes
zero noise in the response, as well as a known prefix for the secret, although a
solution would be to guess the first two characters of the secret, in order to
bootstrap the attack.

From then on, the methodology is the same in general as CRIME's. The attacker
guesses a value, which is then included in the response body along with the
secret and, if correct, it is compressed well with it, resulting in smaller
response length.

BREACH has not yet been fully mitigated, although Gluck, Harris and Prado
proposed various counter measures for the attack. We will investigate these
mitigation techniques in depth in Chapter \ref{ch:mitigation}.
